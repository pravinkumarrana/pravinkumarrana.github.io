%%%%%%%%%%%%%%%%%%%%%%
%
%  Pravin Kumar Rana CV (altacv.cls)
%
%%%%%%%%%%%%%%%%%%%%%%

\documentclass[10pt,a4paper,withhyper]{altacv}
% AltaCV uses the fontawesome5 and simpleicons packages.
% Change the page layout if you need to
\geometry{left=1.25cm,right=1.25cm,top=1.5cm,bottom=1.5cm,columnsep=1.25cm}

% Hyphenation limits
\usepackage[none]{hyphenat}
\hyphenpenalty=5000
\exhyphenpenalty=5000
\lefthyphenmin=2
\righthyphenmin=2
\usepackage{ragged2e}
\RaggedRight
\usepackage{microtype}

% The paracol package lets you typeset columns of text in parallel
\usepackage{paracol}

% For academic icons
\usepackage{academicons} 

% Change the font if you want to, depending on whether
% you're using pdflatex or xelatex/lualatex
% WHEN COMPILING WITH XELATEX PLEASE USE
% xelatex -shell-escape -output-driver="xdvipdfmx -z 0" mmayer.tex
\iftutex
  % If using xelatex or lualatex:
   \setmainfont{Lato}
\else
  % If using pdflatex:
   \usepackage[default]{lato}
\fi
\usepackage{lmodern} \renewcommand{\familydefault}{\sfdefault}

% Change the colours if you want to
\definecolor{SteelBlue}{HTML}{4682B4}
\definecolor{VividBlue}{HTML}{2A7AE2}
\definecolor{VividPurple}{HTML}{3E0097}
\definecolor{SlateGrey}{HTML}{1A4E9E}
\definecolor{LightGrey}{HTML}{333333}

% \colorlet{name}{black}
% \colorlet{tagline}{PastelRed}
\colorlet{heading}{SlateGrey}
\colorlet{headingrule}{SlateGrey}
\colorlet{subheading}{SlateGrey}
\colorlet{accent}{SlateGrey}
\colorlet{emphasis}{SlateGrey}
\colorlet{body}{LightGrey}

% Highligh href links
\newcommand{\hhref}[2]{\textcolor{VividBlue}{\href{#1}{#2}}}

% Change some fonts, if necessary
% \renewcommand{\namefont}{\Huge\rmfamily\bfseries}
% \renewcommand{\personalinfofont}{\footnotesize}
% \renewcommand{\cvsectionfont}{\LARGE\rmfamily\bfseries}
% \renewcommand{\cvsubsectionfont}{\large\bfseries}

% Change the bullets for itemize and rating marker
% for \cvskill if you want to
\renewcommand{\cvItemMarker}{{\small\textbullet}}
\renewcommand{\cvRatingMarker}{\faCircle}
% ...and the markers for the date/location for \cvevent
\renewcommand{\cvDateMarker}{\faCalendar*[regular]}
\renewcommand{\cvLocationMarker}{\faMapMarker*}

% If your CV/résumé is in a language other than English,
% then you probably want to change these so that when you
% copy-paste from the PDF or run pdftotext, the location
% and date marker icons for \cvevent will paste as correct
% translations. For example Spanish:
% \renewcommand{\locationname}{Ubicación}
% \renewcommand{\datename}{Fecha}

% Use (and optionally edit if necessary) this .tex if you
% want an originally numerical reference style like IEEE
% for your publication list
\usepackage[backend=biber,maxnames=99, style=ieee]{biblatex}
%% For removing numbering entirely when using a numeric style
\setlength{\bibhang}{1.25em}
% \DeclareFieldFormat{labelnumberwidth}{\makebox[\bibhang][l]{\itemmarker}}
% \setlength{\biblabelsep}{0pt}

% Define a mock \cvsubsection if your document class doesn't provide it
% (e.g., if you are using 'article' class)
\providecommand{\cvsubsection}[1]{\section*{#1}}

\defbibheading{pubtype}{\cvsubsection{#1}}
\renewcommand{\bibsetup}{\vspace*{-\baselineskip}}
\AtEveryBibitem{%
    \iffieldundef{doi}{}{\clearfield{url}}%
    % Add other custom clearing rules if needed, e.g.,
    % \clearfield{issn}%
    % \clearfield{isbn}%
}


% bib files for publications
\addbibresource{../bibliography/patent-issued.bib}
\addbibresource{../bibliography/patent-published.bib}
\addbibresource{../bibliography/refreed-journals.bib}
\addbibresource{../bibliography/refreed-conferences.bib}
\addbibresource{../bibliography/conference-abstracts.bib}
\addbibresource{../bibliography/preprints.bib}
\addbibresource{../bibliography/tech-reports.bib}
\addbibresource{../bibliography/theses.bib}

% custom google scholar link 
% \NewInfoField{googlescholar}{~\aiGoogleScholar~}[https://scholar.google.com/citations?user=] 
 \newcommand{\customnewline}{\newline\vspace{0.125em}}
 
% start cv
\begin{document}
\name{Pravin Kumar Rana}
% \tagline{} %Next-Gen Technology Professional} 
% You can add multiple photos on the left or right
% \photoR{2.5cm}{mmayer-wikipedia-cc-by-2_0}
% \photoL{2cm}{Yacht_High,Suitcase_High}
\personalinfo{
% \textcolor{VividBlue}%\hhref{https://www.algopra.com}{www.algopra.com} |  \hhref{mailto:pravinkumarrana@gmail.com}{pravinkumarrana@gmail.com} | \hhref{tel:+46738790621}{+46 738-790-621} | Stockholm, Sweden} 
   % \textcolor{VividBlue}{
        % Not all of these are required!
        % You can add your own with \printinfo{symbol}{detail}
        \homepage{algopra.com}\hspace{-1.75em}
        \email{pravinkumarrana@gmail.com}\hspace{-1.75em}
        \linkedin{pravinkumarrana}\hspace{-1.75em}
        \phone{+46 738-790-621}\hspace{-1.75em}\location{Stockholm, Sweden}
        % \mailaddress{Address, Street, 00000 County}
        % \twitter{@marissamayer}
        % \xtwitter{@marissamayer}
        % \googlescholar{9dIuZQcAAAAJ}
        % \github{github.com/mmayer} % I'm just making this up though.
        % \orcid{0000-0000-0000-0000} % Obviously making this up too.
        % You can add your own arbitrary detail with
        % \printinfo{symbol}{detail}[optional hyperlink prefix]
        % \printinfo{\faPaw}{Hey ho!}
        % Or you can declare your own field with
        % \NewInfoFiled{fieldname}{symbol}[optional hyperlink prefix] and use it:
        % \NewInfoField{gitlab}{\faGitlab}[https://gitlab.com/]
        % \gitlab{your_id}
        %
        % For services and platforms like Mastodon where there isn't a
        % straightforward relation between the user ID/nickname and the hyperlink,
        % you can use \printinfo directly e.g.
        % \printinfo{\faMastodon}{@username@instace}[https://instance.url/@username]
        % But if you absolutely want to create new dedicated info fields for
        % such platforms, then use \NewInfoField* with a star:
        % \NewInfoField*{mastodon}{\faMastodon}
        % then you can use \mastodon, with TWO arguments where the 2nd argument is
        % the full hyperlink.
        % \mastodon{@username@instance}{https://instance.url/@username}
%}
}

\makecvheader

% Depending on your tastes, you may want to make fonts of itemize environments slightly smaller
% \AtBeginEnvironment{itemize}{\small}

% Set the left/right column width ratio to 6:4.
\columnratio{0.62}

% Start a 2-column paracol. Both the left and right columns will automatically
% break across pages if things get too long.
\begin{paracol}{2}

\cvsection{Experience}

\cvevent{Product Owner and Project Manager -- XR}{\hhref{https://www.tobii.com}{Tobii AB}}{Aug 2020 -- Present}{Stockholm, Sweden} %Aug 28, 2020
\begin{itemize}
\item Lead 12+ engineers across system, hardware, firmware, and algorithm, delivering next-gen ML eye-tracking platform for XR and smart eyewear, now deployed in \hhref{https://pfdm.ai/}{Play for Dream MR} and \hhref{https://www.tobii.com/products/eye-trackers/wearables/tobii-glasses-x}{Glasses X}
\item Owned full product lifecycle of Tobii’s first XR dual-camera platform, powering XR development at leading firms including \hhref{https://dl.acm.org/doi/10.1145/3588037.3595389}{Meta}
\item Ensured program success by maintaining roadmap and delivery momentum, coordinating global teams and stakeholders during Program Manager’s absence
\item Secured 15+ multimillion‑SEK programs by tailoring platform capabilities, driving revenue growth and meeting evolving client needs
\item Drove 80\% client retention through on‑scope, on‑time delivery, fostering long‑term partnerships via strategic planning and engagement
\item Shared roadmap progress and strategic updates at company-wide all-hands, aligning global leadership and staff on XR priorities
\end{itemize}

\divider

\cvevent{Senior Algorithm Developer -- XR}{\hhref{https://www.tobii.com}{Tobii AB}}{Mar 2016 -- Aug 2020}{Stockholm, Sweden}
\begin{itemize} 
\item Oversaw OEM-specific algorithm integration into 7+ commercial headsets, including \hhref{https://www.vive.com/hk/product/vive-pro-eye/}{HTC Vive Pro Eye}, \hhref{https://www.xrtoday.com/virtual-reality/sony-secures-tobii-eye-tracking-for-psvr-2/}{Sony PlayStation VR2}, and \hhref{https://www.picoxr.com/global/products/neo3-pro-eye/}{Pico Neo 3 Pro Eye}, driving adoption across leading OEMs% 1. Sony PS VR2, 2 HTC Vive, 3 Pico 2, 4 Pico 3, 5 StarVR, 6 PimaxVR, 7 HP, 8 Qualcomm ref designs
\item Served 6+ algorithm engineers as Tech Lead and Scrum Master, delivering Tobii’s VR eye‑tracking platforms with 95\% user coverage and early traction for \hhref{https://news.cision.com/tobii-tech/r/tobii-releases-eye-tracking-vr-development-kit-for-htc-vive,c2275187}{Tobii VR Dev Kit}
\item Led development of scalable optical calibration algorithm for the VR platforms, enabling mass production at \hhref{https://www.player.one/psvr-2-tobii-eye-tracking-confirmed-148486}{top-tier OEMs} with defect rates below 50 parts per million% (PPM).
\item Bridged engineering, product, and sales to translate client requirements into production‑ready solutions, accelerating headset launches across multiple OEMs
%\item Orchestrated Agile workflows across sprints, backlog grooming, and release planning with product owners
%\item Orchestrated sprints and backlog with Product Owners, leveraging team capacity and sprint velocity insights to consistently achieve \~90\% sprint goal completion
\item Orchestrated sprints and backlog with Product Owners, leveraging sprint throughput insights to consistently achieve \~90\% sprint goals and accelerate release cadence
\end{itemize}

\divider

\cvevent{Senior Algorithm Developer -- Peripheral Devices}{\hhref{https://www.tobii.com}{Tobii AB}}{Mar 2014 -- Feb 2016}{Stockholm, Sweden}% Mar 31, 2014
\begin{itemize} 
\item Developed embedded algorithms for \hhref{https://news.cision.com/tobii-ab/r/tobii-unveils-next-generation-eye-tracking-platform-targeting-consumer-devices,c9847938}{Tobii’s IS4 eye-tracking platform}, deployed in \hhref{https://www.dell.com/en-us/shop/dell-laptops/sr/laptops/alienware-laptops}{Dell} and \hhref{https://www.acer.com/us-en/laptops/gaming}{Acer} laptops and medical devices including \hhref{https://store.prc-saltillo.com/category/accent}{PRC-Saltillo Accent}, expanding market adoption
\item Built real-time, user-position-aware camera control algorithm, making the IS4 platform \hhref{https://uk.pcmag.com/software/73918/log-in-to-windows-hello-with-tobii-eye-trackers}{Windows Hello–compliant} for secure biometric login
\item Co-developed a subpixel-accurate optical calibration solution, achieving 99.8\% first-pass mass production yield across 1M+ IS4 units
\end{itemize}

\divider

\cvevent{Graduate Researcher -- School of Electrical Engineering}{\hhref{https://www.kth.se/}{KTH Royal Institute of Technology}}{Sep 2008 -- Mar 2014}{Stockholm, Sweden} % ACCESS Linnaeus Center, School of Electrical Engineering
\begin{itemize} 
\item Designed novel 3D scene geometry and rendering algorithms, improving view synthesis quality by up to \hhref{https://ieeexplore.ieee.org/abstract/document/7074087}{4 dB in PSNR} and enriching \hhref{https://www.algopra.com/publications/pdfs/unpublished/Rana11_FTV.pdf}{Free-viewpoint TV} immersion
\item Engineered multiview video and depth processing techniques with Ericsson AB, resulting in 2 U.S. patents, 8 peer-reviewed publications, and input to \hhref{https://urn.kb.se/resolve?urn=urn:nbn:se:kth:diva-108327}{MPEG} standards
\item Taught image and video processing and information theory, supervised student projects, and evaluated performance across academic levels
\end{itemize}

\cvsection{Education}

\cvevent{M.Tech\ in Earth System Science and Technology}{\hhref{https://www.iitkgp.ac.in/}{Indian Institute of Technology}}{Jul 2006 -- Jun 2008}{Kharagpur, India}
\begin{itemize}[label={}]
	\item  \textbf{Specialization}: Satellite Oceanography
	\item  \textbf{Cumulative GPA}: 9.53/10.00
\end{itemize}

\divider

\cvevent{M.Sc.\ in Physics}{\hhref{https://ranchiuniversity.ac.in/}{Ranchi University}}{Sep 2002 -- Jun 2004}{Ranchi, India}
\begin{itemize}[label={}]
	\item \textbf{Specialization}: Electronics and Communication
	\item  \textbf{First Class Honours}: 71.81\%
\end{itemize}

\cvsection{Certifications}

\cvevent{AI Product Manager Specialization}{\hhref{https://coursera.org/share/7e0033786093b4ae3dc07b6e743ff729}{IBM, SkillUp}}{Aug 2025}{Coursera}
\divider 

\cvevent{Project Management Specialization}{\hhref{https://coursera.org/share/d1adf1a1fef03d333370c83872a8f46c}{Google}}{Aug 2025}{Coursera}
\divider 

\cvevent{Deep Learning Specialization}{\hhref{https://coursera.org/share/9910b73a9a1b895decd9536cdf4063d7}{DeepLearning.AI}}{Mar 2018}{Coursera}
\divider 

\cvevent{Certified ScrumMaster (CSM)}{\hhref{https://www.algopra.com/source/certificates/ScrumAlliance/ScrumAlliance_CSM_Certificate_20171119.pdf}{Scrum Alliance}}{Aug 2017, Expired: Aug 2019}{Stockholm, Sweden}

%\divider 

%\cvevent{\hhref{https://www.coursera.org/account/accomplishments/verify/48PZC9AAWZYH}{Programming for Everybody}}{\h\hhref{https://www.coursera.org/specializations/python}{University of Michigan}}{Aug 2016 --  Aug 2016}{Coursera}

% footnote
\vfill
{\footnotesize Updated \today, Stockholm, Sweden}

% use ONLY \newpage if you want to force a page break for
% ONLY the currentc column
\newpage

% Switch to the right column. This will now automatically move to the second
% page if the content is too long.
\vfill
\switchcolumn
  
\cvsection{Summary}

Technology professional with 15+ years at the intersection of embedded systems, computer vision, ML, and XR. Delivered integrated hardware-software platforms and led agile cross-functional teams. Architected scalable solutions adopted by global tech leaders like Meta, ByteDance, and Sony. Skilled in shaping product vision, aligning stakeholders, and translating complex challenges into user-centric technologies. Adept at stakeholder navigation, agile execution, and fostering collaboration across technical and business domains. Passionate about advancing innovations that empower users and deliver meaningful impact.

\cvsection{Competencies}
%
\cvachievement{\faIndustry}{Management}{Strategic Prioritization, Stakeholder Engagement, Product Ownership, Project Management,  Backlog Coordination, KPI-Driven Decisions, Program Oversight}

\divider

\cvachievement{\faUsers}{Leadership}{Agile, Scrum, SAFe, Release Planning, Workflow Optimization, Technical Mentorship, Continuous Improvement, Cross-Functional Collaboration}

\divider

\cvachievement{\faTools}{Technical Expertise}{\textbf{Domain}: Extended Reality (XR: AR/VR/MR), Machine Learning, Eye Tracking, Computer Vision, Optical Systems, Embedded Systems \customnewline \textbf{Programming}: C++, Python  \customnewline \textbf{Tools}: Git, JIRA, Confluence, Power BI, Microsoft 365}

\cvsection{Innovations}
%
\cvachievement{\faFileSignature}{U.S. Patents}{\textbf{Granted Patents}: 20 \customnewline \textbf{Pending Patents}: 4  \customnewline \textbf{View Full Portfolio}: \hhref{https://patents.google.com/?inventor=Pravin+Rana&country=US&status=GRANT&dups=language}{Google Patents}\customnewline \textbf{Innovation Domains}: XR Platforms, Eye Tracking, 3D Imaging, Multiview Video, Optical System}

\divider

\cvachievement{\faBookOpen}{Peer-Reviewed Publications}{\textbf{Journal Articles}: 3 \customnewline \textbf{Conference Papers}: 7 \customnewline \textbf{View Full List}: \hhref{https://scholar.google.se/citations?user=9dIuZQcAAAAJ&hl=en}{Google Scholar} \customnewline \textbf{Research Areas}: Multiview Video, Machine Learning, Computer Vision, Depth Estimation}

\cvsection{Achievements}

\cvachievement{\faAward}{Tobii Star Performer Award}{Earned performance-based RSUs for consistently driving high-impact results in Tobii's XR program}

\divider

\cvachievement{\faLightbulb}{Tobii Top IP Contributor}{Credited with 18 U.S. patents powering headsets across leading brands, such as Sony, HTC, and ByteDance}

\divider

\cvachievement{\faRocket}{Tobii Dragon's Den Finalist}{Honored with second place for pitching a novel eye-tracking algorithm in a company-wide challenge}

\divider

\cvachievement{\faUniversity}{National Scholarship Recipient}{Awarded merit-based scholarship by the Government of India for studies at Indian Institute of Technology}

\divider

\cvachievement{\faChartLine}{India \hhref{https://en.wikipedia.org/wiki/Graduate_Aptitude_Test_in_Engineering}{GATE} Top Ranker}{Ranked 254 out of 4,904 in India's national entrance exam  in Physics for Master’s programs}

\divider

\cvachievement{\faMedal}{University Gold Medalist}{Graduated top of class in M.Sc. Physics from Ranchi University, earning highest academic honors}

\divider

\cvachievement{\faStar}{TIME Best Inventions}{Eye-tracking work featured in TIME’s via \hhref{https://time.com/collection/best-inventions-2019/5733087/htc-vive-pro-eye/}{HTC Vive Pro Eye (2019)} and \hhref{https://time.com/collection/best-inventions-2020/5911406/pico-neo-2-eye/}{Pico Neo 2 Eye (2020)}}

\divider

\cvachievement{\faMicrophone}{Public Communicator}{Presented on research, emerging tech, and product strategy at 12+ global conferences, Tobii internal tech forums, and company-wide all-hands}


\cvsection{Languages}

\cvskill{English}{4.5}

\divider

\cvskill{Hindi}{5}

%\cvsection{Swedish Visa Status}
%\cvsection{Referees}
%\cvref{name}{email}{mailing address}
%{Address Line 1\\Address line 2}

\end{paracol}


%\cvsection{Appendix A: Patents}
%
%% patent grants
%\begin{refsection}[../bibliography/patent-issued.bib]
%  \nocite{*}
%  \printbibliography[heading=pubtype,title={\printinfo{\faCertificate}{US Patent Grants}},type=inproceedings]
%\end{refsection}
%
%\divider
%% pending patents
%\begin{refsection}[../bibliography/patent-published.bib]
%  \nocite{*}
%  \printbibliography[heading=pubtype,title={\printinfo{\textcolor{gray}{\faCertificate}}{US Patent Applications}},type=inproceedings]
%\end{refsection}
%
%\newpage
%\cvsection{Appendix B: Publications}
%
%% articles
%\begin{refsection}[../bibliography/refreed-journals.bib]
%  \nocite{*}
%  \printbibliography[heading=pubtype,title={\printinfo{\faFile*[regular]}{Refreed Journal Articles}},type=article]
%\end{refsection}
%
%\divider
%
%% proceedings
%\begin{refsection}[../bibliography/refreed-conferences.bib]
%  \nocite{*}
%  \printbibliography[heading=pubtype,title={\printinfo{\faUsers~}{Refreed Conference Proceedings}},type=inproceedings]
%\end{refsection}
%
%\divider
%
%% preprints
%\begin{refsection}[../bibliography/preprints.bib]
%  \nocite{*}
%  \printbibliography[heading=pubtype,title={\printinfo{\faFile*[regular]}{Preprints}},type=article]
%\end{refsection}
%
%\divider
%
%%tech reports
%\begin{refsection}[../bibliography/tech-reports.bib]
%  \nocite{*}
%  \printbibliography[heading=pubtype,title={\printinfo{\faFile}{Technical Reports}},type=report]
%\end{refsection}
%
%\divider
%
%%conference abstracts
%\begin{refsection}[../bibliography/conference-abstracts.bib]
%  \nocite{*}
%  \printbibliography[heading=pubtype,title={\printinfo{\faUsers~}{Conference Abstracts}},type=inproceedings]
%\end{refsection}
%
%% footnote
%\vfill
%{\footnotesize Updated \today, Stockholm, Sweden}


%\newpage
%\cvsection{Appendix C: Speaking Engagements}
%
%\begin{enumerate}[label={[\arabic*]}]
% \item \textbf{\textit{Introduced Tobii XR5-ML plaform}}, All-Hands Meeting, Tobii AB, Stockholm, Sweden, 2025
%  \item \textbf{\textit{Eye tracking in the 5G era}}, Tobii Develop Beyond, Stockholm, Sweden, 2019
%  \item \textbf{\textit{Tobii eye tracking VR development kit -- Demo session}}, ACCESS Workshop, Stockholm, Sweden, 2017
%  \item \textbf{\textit{Stereo vision–based distance estimation in Tobii IS4 eye-tracking platform}}, Tobii Develop Beyond, Stockholm, Sweden, 2015
%  \item \textbf{\textit{Statistical methods for inter-view depth enhancement}}, 3DTV-Con, Budapest, Hungary, 2014
%  \item \textbf{\textit{Multiview depth map enhancement by variational Bayes inference estimation of Dirichlet mixture models}}, IEEE ICASSP, Vancouver, Canada, 2013
%  \item \textbf{\textit{A variational Bayesian inference framework for multiview depth image enhancement}}, IEEE ISM, Irvine, USA, 2012
%  \item \textbf{\textit{Denoising of volumetric depth confidence for view rendering}}, 3DTV-Con, Zurich, Switzerland, 2012
%  \item \textbf{\textit{Depth pixel clustering for consistency testing of multiview depth}}, EUSIPCO, Bucharest, Romania, 2012
%  \item \textbf{\textit{View interpolation with structured depth from multiview video}}, EUSIPCO, Barcelona, Spain, 2011
%  \item \textbf{\textit{Depth consistency testing for improved view interpolation}}, IEEE MMSP, Saint Malo, France, 2010
%  \item \textbf{\textit{Prediction of Antarctic sea ice edge using artificial intelligence}}, SCAR/IASC IPY Open Science Conference, St. Petersburg, Russia, 2008
%  \item \textbf{\textit{Prediction of sea ice edge by using image processing techniques}}, DTU, Risø, Denmark, 2008
%\end{enumerate}
%% footnote
%\vfill
%{\footnotesize Updated \today, Stockholm, Sweden}

\end{document}

