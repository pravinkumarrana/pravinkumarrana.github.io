\documentclass[8pt]{beamer}
\mode<presentation>
{
  \usetheme{boxes}
  \setbeamercovered{transparent}
}
\usepackage{subfig}
\usepackage{amsmath,epsfig, psfrag, hyperref}
\usepackage{amssymb, algorithm, algorithmicx}
\usepackage{algpseudocode}
\usepackage{caption}
\usepackage[english]{babel}
\usepackage[latin1]{inputenc}
\usepackage{times}
\usepackage[T1]{fontenc}
\title[Project Assignment \#1] {\textbf{Project Assignment \#1}}
\author{\textbf{FEM1230 Estimation Theory}}



\begin{document}

\begin{frame}
  \titlepage
  \begin{center}
  \textcolor[rgb]{0.29,0.29,0.29}{\textbf{Pravin Kumar Rana}} \textcolor[rgb]{0.29,0.29,0.29}{\\SIP, School of Electrical Engineering\\ Kungliga Tekniska h�gskolan \\ SE-$100 44$ Stockholm}
  \end{center}
\end{frame}

\begin{frame}{Outline}
\begin{block}{}
\begin{itemize}
  \item Signal Model\vspace{3mm}
  \item Cramer-Rao Lower Bound\vspace{3mm}
  \item Maximum Likelihood Estimation\vspace{3mm}
  \item Simulation Results
\end{itemize}

\end{block}

\end{frame}

%Measurements and Data Model
\section{Signal Model}
\begin{frame}{Signal Model}
\begin{itemize}
  \item Assume a signal model
    \begin{equation}
    y[n] = s[n] + w[n] \quad n = 1, 2, \ldots, N-1
    \end{equation}
    where the signal of interest $s[n]$ modeled by
    \begin{equation}
    s[n; A, B, C, \omega] = A \cos(\omega n) + B \sin(\omega n) + C
    \end{equation}
    with A, B, C as unknown constants and $w[n]$ the additive zero mean noise.\vspace{3mm}
  \item Angular frequency $\omega$ may be known, or not, leading to models with three or four unknown parameters, respectively.\vspace{3mm}
  \item Consider the leakage parameter $L$, given as
    \begin{equation}
        L = \frac{2C^2}{A^2+B^2}.
    \end{equation}
\end{itemize}
\end{frame}

% Cramer-Rao Lower Bound
\section{Cramer-Rao Lower Bound}
\begin{frame}{Cramer-Rao Lower Bound}
\begin{itemize}
\item Regularity conditions hold
\begin{itemize}
  \item The Gaussian probability density function for the signal model
    \begin{equation}
    {  \displaystyle{p(\mathbf{y},\mathbf{\theta})=\frac{1}{(2\pi \sigma^2)^{\frac{N}{2}}}\exp\left[\frac{-1}{2\sigma^2}(\mathbf{y}-\mathbf{s}(\mathbf{\theta}))^{T}(\mathbf{y}-\mathbf{s}(\mathbf{\theta}))\right],}}
    \end{equation}
    satisfies the \emph{regularity} conditions for all $\theta$, i.e.,
    \begin{align}\nonumber
    \displaystyle{\mathrm{E}\left[ \frac{\partial \ln p(\mathbf{y};\mathbf{\theta)}} {\partial \mathbf{\theta}}\right]} = 0 \quad \forall \theta
    \end{align}
     where  $\mathbf{y} = [y[0]\ldots y[N-1]]^{T}$ and  $\mathbf{s}(\mathbf{\theta})=[s[0;\mathbf{\theta} ]\ldots s[N-1;\mathbf{\theta}]]^{T}$.
\end{itemize}
\item Calculate the Fisher information matrix
\begin{itemize}
  \item[] \begin{align}\label{FM1}
        \mathbf{I}={\left[
        \begin{array}{cccc}
        \frac{N}{2\sigma^2} & 0 & 0 & \frac{B N(N-1)}{4\sigma^2} \\
        0 & \frac{N}{2\sigma^2} & 0 & -\frac{A N(N-1)}{4\sigma^2} \\
        0 & 0 & \frac{N}{\sigma^2} & 0 \\
        \frac{B N(N-1)}{4\sigma^2} & -\frac{A N(N-1)}{4\sigma^2} & 0 & \frac{(A^2+B^2)(2N^3-3N^2+N)}{12\sigma^2}
        \end{array}
    \right]}
    \end{align}
    where
    \begin{eqnarray}\nonumber
        [\mathbf{I}(\mathbf{\theta})]_{i,j} &=& \frac{1}{\sigma^2}\sum_{n=0}^{N-1}\frac{\partial s[n; \theta]}{\partial \theta_{i}}\frac{\partial s[n; \theta]}{\partial\theta_{j}}.
    \end{eqnarray}
    \end{itemize}
\end{itemize}
 \end{frame}

%Asymptotic Expression
\begin{frame}{Cramer-Rao Lower Bound}
 \begin{itemize}
    \item The asymptotic expression CRLB is given by
    \begin{itemize}
    \item[]\begin{align}\label{CRB1}
        CRLB(\mathbf{\theta})&=\mathbf{I}(\mathbf{\theta})^{-1}\approx 2\sigma^{2}\mathbf{I}_{1}^{-1},
    \end{align}
    where
    \small{\begin{align}
            \mathbf{I}_{1}&=\left[
            \begin{array}{cccc}
            N & 0 & 0 & \frac{BN^2}{2} \\
            0 & N & 0 & -\frac{AN^2}{2} \\
            0 & 0 & 2N & 0 \\
            \frac{BN^2}{2} & -\frac{AN^2}{2} & 0 & \frac{(A^2+B^2)N^3}{3}
            \end{array}
            \right].
            \end{align}}


 \end{itemize}
 \end{itemize}
\end{frame}



%\subsection{Cramer-Rao Bound for Different Parameter Model}
\begin{frame}{Cramer-Rao Lower Bound}
\begin{itemize}
  \item For the three parameter model% (known frequency):
    \begin{itemize}
    \item[]  \vspace{-3mm}
    \begin{align}
    var(\hat{A}) &\geq CRB(A)\approx \frac{2\sigma^2}{N}\\
    var(\hat{B}) &\geq CRB(B)\approx \frac{2\sigma^2}{N}\\
    var(\hat{C}) &\geq CRB(C)\approx \frac{\sigma^2}{N}
    \end{align}
    \end{itemize}
  \item For the four parameter model% (unknown frequency):
  \begin{itemize}
  \item[] \vspace{-3mm}
   \begin{align}
    var(\hat{A})      &\geq CRB(A)     \approx  \frac{2\sigma^{2}}{N}(1+\frac{3B^2}{\alpha^2})\\
    var(\hat{B})      &\geq CRB(B)     \approx  \frac{2\sigma^{2}}{N}(1+\frac{3A^2}{\alpha^2})\\
    var(\hat{C})      &\geq CRB(C)     \approx  \frac{\sigma^2}{N}\\
    var(\hat{\omega}) &\geq CRB(\omega)\approx  \frac{24\sigma^2}{\alpha^2N^3}
    \end{align}
    \end{itemize}
  \item For the Leakage parameter%, by using the transformation of parameters,
  \begin{itemize}
  \item[]   \vspace{-3mm}
    \begin{align}
    var(\hat{L})      &\geq CRB(L) = {2\sigma^2}\left(\frac{16C^4}{\alpha^{6}N}+\frac{8C^2}{\alpha^{4}N}\right)
     %CRB(L) =\left[\frac{\partial L}{\partial \mathbf{\theta}}\right] CRB(\mathbf{\theta})\left[\frac{\partial L}{\partial \mathbf{\theta}}\right]^{T} = {2\sigma^2}\left(\frac{16C^4}{\alpha^{6}N}+\frac{8C^2}{\alpha^{4}N}\right)
    \end{align}
    \end{itemize}
\end{itemize}
\end{frame}

%
%mle
\section{Maximum Likelihood Estimation}
\subsection{Motivation}
\begin{frame}{Maximum Likelihood Estimation(MLE)}

\begin{block}{Motivation}
\begin{itemize}
    \item MLE is \emph{asymptotically optimal} because of its asymptotic properties of being unbiased, achieving the CRLB and having a Gaussian PDF.\vspace{3mm}
    \item MLE is efficient for the estimation of L and the MLE of the unknown vector parameter $\theta$ is asymptotically distributed according to
    \begin{align}
    \hat{\theta} & \overset{a}{\thicksim} \mathcal{N}(\mathbf{\theta}, \mathbf{I}^{-1}(\mathbf{\theta})),
    \end{align} because PDF $p(y;\theta)$ hold \emph{regularity conditions} (\textbf{{Theorem 7.3 [2]}}). %where $\mathbf{I}(\mathbf{\theta})$ is the Fisher information matrix evaluated at the true value of the unknown parameter $\mathbf{\theta}$
    \end{itemize}
\end{block}
\end{frame}

\subsection{Three Parameter Model}
\begin{frame}{Three Parameter Model}
\begin{itemize}
  \item ${s[n,\mathbf{\theta}]}$ is linear in parameters $\mathbf{\theta}$, i.e.,
\begin{align}\label{eq:22}
\mathbf{s} &= \mathbf{H}\mathbf{\theta}
\end{align}where $\theta=[A~ B~ C]^{T}$ and
\begin{align}\label{H1}
\mathbf{H} &= \left[\begin{array}{ccc}
               1                    & 0                 & 1 \\
               \cos(\omega)         & \sin(\omega)      & 1 \\
               \vdots               & \vdots            & \vdots \\
               \cos(\omega (N-1))   & \sin(\omega(N-1)) & 1\\
               \end{array}\right].
\end{align}
\item Under Gaussian assumption, the MLE of $\mathbf{\theta}$ is found by minimizing
\begin{align}\label{eq:22}
J(\mathbf{\theta}) &=  (\mathbf{y}-\mathbf{H\theta})^{T}(\mathbf{y}-\mathbf{H\theta}).
\end{align}
\item The minimizing solution $\hat{\mathbf{\theta}} = [\hat{A}~ \hat{B}~ \hat{C}]^{T}$ is
\begin{align}\label{UF1}
\hat{\mathbf{\theta}} &= (\mathbf{H}^{T}\mathbf{H})^{-1}\mathbf{H}^{T}\mathbf{y}.
\end{align}
\end{itemize}
\end{frame}

\subsection{Four Parameter Model}
\begin{frame}{Four Parameter Model}
\begin{itemize}
  \item ${s[n,\mathbf{\theta}]}$ is nonlinear in parameters $\mathbf{\theta}$.
  \item  Assuming $\hat{\omega}_{i}$ is an estimate in iteration $i$ of ${\omega}$, a Taylor approx. around $\hat{\omega}_{i}$ gives
\begin{align}\label{APU1}
\cos(\omega n)&\approx \cos(\hat{\omega}_{i} n)-n\cos(\hat{\omega}_{i} n)\cdot\Delta\omega_{i}\\
\sin(\omega n)&\approx \sin(\hat{\omega}_{i} n)+n\cos(\hat{\omega}_{i} n)\cdot\Delta\omega_{i}.
\end{align}
This gives
\begin{align}\label{APU2}
s[n,\mathbf{\theta}]&=A\cos(\hat{\omega}_{i}n)+B\sin(\hat{\omega}_{i} n)+C-An\Delta\hat{\omega}_{i}\sin(\hat{\omega}_{i} n)+Bn\Delta\hat{\omega}_{i}\cos(\hat{\omega}_{i} n)
\end{align} where $\Delta\hat{\omega}_{i}=\omega-\hat{\omega}_{i}$.

\item Using above approximations, we can write
\begin{align}\label{APU4}
\mathbf{s}&=\mathbf{H}_{i}\mathbf{\theta}_{i}\\
J(\mathbf{\theta}) &= (\mathbf{y}-\mathbf{H_{i}\theta_{i}})^{T}(\mathbf{y}-\mathbf{H_{i}\theta_{i}}).
\end{align}where $\mathbf{\theta}_{i}=[A~~B~~C~~\Delta\hat{\omega}_{i}]^T$ and
\begin{align}\nonumber
\footnotesize{\mathbf{H}_{i}=\left[
\begin{array}{cccc}
    \hspace{-.2cm}   1  &\hspace{-.2cm} 0 & \hspace{-.2cm} 1 &  0\\
    \hspace{-.2cm}\cos(\omega_{i}) &\hspace{-.2cm} \sin(\omega_{i}) & \hspace{-.2cm} 1 & \hspace{-.2cm}-A_{i-1}\sin(\omega_{i})+B_{i-1}\cos(\omega_{i})\hspace{-.2cm} \\
    \hspace{-.2cm}\vdots              &\hspace{-.2cm} \vdots              & \hspace{-.2cm} \vdots  & \hspace{-.3cm}\hspace{-.2cm}\vdots\\
    \hspace{-.2cm}\cos(\omega_{i}{(N-1)}) &\hspace{-.2cm} \sin(\omega_{i}{(N-1)}) & \hspace{-.2cm} 1 & \hspace{-.2cm}-A_{i-1}{(N-1)}\sin(\omega_{i}{(N-1)})+B_{i-1}{(N-1)}\cos(\omega_{i}{(N-1))}\nonumber \hspace{-.2cm}
  \end{array}
\right].}
\end{align}
\end{itemize}
\end{frame}

\begin{frame}{Four Parameter Model}
\begin{itemize}
 % \item 1057-1994 IEEE Standard for Digitizing Waveform Recorders. \vspace{3mm}
  \item \textbf{Algorithm I} [IEEE Std 1057-1994 (R2001)]\vspace{3mm}
\begin{algorithmic}[1]
\State Set index $i=0$
\State \textbf{Initialize}: ${ \hat{\omega}}_{0}, \hat{\theta}_{0}=[A_{0} ~B_{0},C_{0}, ~0]^{T}$
\Repeat
\State $i\gets i+1$
\State $\hat{\omega}_{i}\gets \hat{\omega}_{i-1}+\Delta\hat{\omega}_{i-1}$
\State Create $\mathbf{H}_{i}$ %using~\ref{Hi}
\State $\hat{\theta}_{i} \gets (\mathbf{H}_{i}^T\mathbf{H}_{i})^{-1}\mathbf{H}_{i}^{T}\mathbf{y}$
\Until{convergence}\vspace{3mm}
\end{algorithmic}
\item The algorithm iteratively minimize $J$ in order to obtain a new set of estimates of unknown parameters.\vspace{3mm}
\item In each iteration, an updated frequency estimate $\hat{\omega}_{i}$ is obtained based on $\hat{\omega}_{i-1}$ and  $\Delta\hat{\omega}_{i-1}$,  andestimates that also depend on estimated values of A, B, and C.
\end{itemize}
\end{frame}

% Simulation results
\section{Simulation Results}
\begin{frame}{Simulation Results}
\begin{itemize}
  \item The performance of parameter estimations are evaluated by the Monte-Carlo simulations using the empirical mean-squared error criteria,
  \begin{align}
  \small EMSE &=\frac{1}{M}\sum_{m=0}^{M-1} ( \hat{L}_{m}-\mathrm{E} ( \hat{L}_{m}) )
  \end{align} where M is number of independent realization. \vspace{3mm}
  \item In experiment,
   \begin{align}
   M&= 2000\\
   10 \leq N \leq10000\\
   L&\ll(A^2+B^2)\\
   SNR&\gg 20~dB
   \end{align} \vspace{3mm}
  \item An initial estimate of the angular frequency is obtained by using number of sample (N) and an initial guess of sampling frequency (k$\pi$); in experiment k=1000.% for example, counting zero crossings.
\end{itemize}
\end{frame}

\begin{frame}{Signal-to-noise ratio}
\small{\begin{itemize}
  \item By increasing SNR, difference between the EMSE and the CRLB decreases.
  \item By increasing the number of samples, difference between the EMSE and the CRLB further decreased.
\end{itemize}}
\begin{figure}[!t]
    \psfrag{b}[c][c][0.7][0]{Logarithmic Scale (in dB)}
    \psfrag{a}[c][c][0.7][0]{SNR (in dB)}
    \centerline{\psfig{figure=CRLB_&_EMSE_vs_SNR_for_different_N,width=6cm, height=6cm}}
    \caption{The CRLB and the EMSE as the function of SNR.}
	\label{fig:SNR_plot}
\end{figure}
\end{frame}



\begin{frame}{Leakage parameter}
\small{\begin{itemize}
  \item Reducing noise level by decreasing L, increases the difference between the EMSE and the CRLB.
  \item By increasing the number of samples, difference between the EMSE and the CRLB decreases.
\end{itemize}}

\begin{figure}[!t]
    \psfrag{b}[c][c][0.7][0]{Logarithmic Scale (in dB)}
    \psfrag{a}[c][c][0.7][0]{L (in dB)}
    \centerline{\psfig{figure=CRLB_&_EMSE_vs_L_for_different_N,width=6cm, height=6cm}}
    \caption{The CRLB and the EMSE as the function of L.}
	\label{fig:L_plot}
\end{figure}
\end{frame}

\begin{frame}{Number of samples}
\small{\begin{itemize}
  \item Results indicates that the estimator is asymptotically efficient after N > 3000.
\end{itemize}}
\begin{figure}[!t]
    \psfrag{b}[c][c][0.7][0]{Logarithmic Scale (in dB)}
    \psfrag{a}[c][c][0.7][0]{N}
    \centerline{\psfig{figure=CRLB_&_EMSE_vs_N.eps,width=6cm, height=6cm}}
    \caption{The CRLB and the EMSE as the function of N.}
\label{fig:N_plot}
\end{figure}
\end{frame}
%Conclusion
\section{}
\begin{frame}{Conclusion}
\begin{itemize}
  \item The experimental studies revealed that MLE is an asymptotically optimal estimator for the considered signal model.\vspace{3mm}
  \item The difference between estimated values of parameters from both the three and four parameter model are negligible as suggested by the theory.\vspace{3mm}
  \item The convergence properties of estimation depend on the initial estimates.
\end{itemize}
\end{frame}

\end{document}
%\section{Simulation Results}

%\begin{frame}{Simulation Results}
%\begin{itemize}
%  \item The performance of parameter estimations are evaluated by the Monte-Carlo simulations using the empirical mean-squared error criteria,
%  \begin{align}
%  \small EMSE &=\frac{1}{M}\sum_{m=0}^{M-1} ( \hat{L}_{m}-\mathrm{E} ( \hat{L}_{m}) )
%    \end{align}where M is number of independent realization. In experiment, M $= 2000$ with $L\ll(A^2+B^2)$ and $SNR \gg 20$ dB.\vspace{-3mm}
%\end{itemize}
%\begin{figure}[!t]
%    %\centering
%    \raggedright
%    \psfrag{b}[c][c][0.5][0]{Logarithmic Scale (in dB)}
%    \psfrag{a}[c][c][0.5][0]{SNR (in dB)}
%    \subfloat[Signal-to-noise ratio]{\label{fig:snr}{\psfig{figure=CRLB_&_EMSE_vs_SNR_for_different_N,width=4cm, height=4cm}}}
%    \psfrag{a}[c][c][0.5][0]{N}
%    \subfloat[Number of  samples]{\label{fig:sn}{\psfig{figure=CRLB_&_EMSE_vs_N.eps,width=4cm, height=4cm}}}
%    \psfrag{a}[c][c][0.5][0]{L (in dB)}
%    \subfloat[Leakage parameter]{\label{fig:lp}{\psfig{figure=CRLB_&_EMSE_vs_L_for_different_N,width=4cm, height=4cm}}}
%    \caption{Simulation Results}
%\end{figure}
%\end{frame}


%\subsection{Known Frequency}
%\begin{frame}{Three Parameter Model}
%\begin{itemize}
%  \item For three parameter model $\theta = [A~ B~ C]^T$,  the corresponding Fisher matrix is given by the the upper left $3\times3$ submatrix of $\frac{1}{2\sigma^2}\mathbf{I}_{1}$, which is a diagonal matrix for large N.
%  \item The inverse of the submatrix and the diagonal elements of the inverse gives the lower bound on the variance of the estimates,
%  \begin{align}
%    var(\hat{A}) &\geq CRB(A)\approx \frac{2\sigma^2}{N}\\
%    var(\hat{B}) &\geq CRB(B)\approx \frac{2\sigma^2}{N}\\
%    var(\hat{C}) &\geq CRB(C)\approx \frac{\sigma^2}{N}
%\end{align}
%\end{itemize}
%\end{frame}
% %\begin{itemize}
%    \item For large N (as $N\rightarrow~\infty$), we can write the matrix $\mathbf{I(\theta)}$ as
%         \begin{align}
%            \mathbf{I(\theta)}&=\frac{1}{2\sigma^2}(\mathbf{I}_{1}+\mathbf{I}_{2}),
%        \end{align}
%        where
%        \small{\begin{align}
%            \mathbf{I}_{1}&=\left[
%            \begin{array}{cccc}
%            N & 0 & 0 & \frac{BN^2}{2} \\
%            0 & N & 0 & -\frac{AN^2}{2} \\
%            0 & 0 & 2N & 0 \\
%            \frac{BN^2}{2} & -\frac{AN^2}{2} & 0 & \frac{(A^2+B^2)N^3}{3}
%            \end{array}
%            \right]
%            \end{align}} and
%        \small{\begin{align}
%\mathbf{I}_{2}&=\left[
%  \begin{array}{cccc}
%    \mathcal{O}(1) & \mathcal{O}(1) & \mathcal{O}(1) & \mathcal{O}(N) \\
%    \mathcal{O}(1) & \mathcal{O}(1) & \mathcal{O}(1) & \mathcal{O}(N) \\
%    \mathcal{O}(1) & \mathcal{O}(1) & \mathcal{O}(1) & \mathcal{O}(N) \\
%    \mathcal{O}(N) & \mathcal{O}(N) & \mathcal{O}(N) & \mathcal{O}(N^2)
%  \end{array}
%\right],
%\end{align}}
%where  $\mathcal{O}(1)$ and $\mathcal{O}(N)$ represents a bounded quantity and a quantity that is asymptotically linear in $N$, respectively.
%   \item The asymptotic expression CRLB is given by,% as described in Appendix B of~\cite{3},
%    \begin{align}\label{CRB1}
%        CRLB(\mathbf{\theta})&=\mathbf{I}(\mathbf{\theta})^{-1}=2\sigma^{2}(\mathbf{I}_{1}+\mathbf{I}_{2})^{-1}\approx 2\sigma^{2}\mathbf{I}_{1}^{-1},
%    \end{align}
%     \end{itemize}
%
%
%\subsection{Unknown Frequency}
%\begin{frame}{Four Parameter Model}
%\begin{itemize}
%  \item For four parameter model $\theta=[A~B~C~\omega]^{T}$, the inverse of the Fisher matrix is
%\begin{eqnarray}
%\mathbf{I}_{1}^{-1}=\left[\
%  \begin{array}{cccc}
%     \frac{1}{N}(1+\frac{3B^2}{\alpha^2}) & -\frac{3AB}{\alpha^2 N}               & 0            &  \frac{6B}{\alpha^2 N^2} \\
%    -\frac{3AB}{\alpha^2 N}               &  \frac{1}{N}(1+\frac{3A^2}{\alpha^2}) & 0            & -\frac{6A}{\alpha^2 N^2} \\
%     0                                    &  0                                    & \frac{1}{2N} &  0 \\
%    \frac{6B}{\alpha^2 N^2}               & -\frac{6A}{\alpha^2 N^2}              & 0            &  \frac{12 }{\alpha^2 N^3}\\
%  \end{array}
%\right]
%\end{eqnarray}
%  \item The CRLB on the parameters are
%  \begin{align}
%    var(\hat{A})      &\geq CRB(A)     \approx  \frac{2\sigma^{2}}{N}(1+\frac{3B^2}{\alpha^2})\\
%    var(\hat{B})      &\geq CRB(B)     \approx  \frac{2\sigma^{2}}{N}(1+\frac{3A^2}{\alpha^2})\\
%    var(\hat{C})      &\geq CRB(C)     \approx  \frac{\sigma^2}{N}\\
%    var(\hat{\omega}) &\geq CRB(\omega)\approx  \frac{24\sigma^2}{\alpha^2N^3}
%    \end{align}
%\end{itemize}
%\end{frame}
%

%\subsection{Cramer-Rao Bound for Different Parameter Model}
%\begin{frame}{Cramer-Rao bound}
%\begin{itemize}
%  \item The CRLB for the three parameter model (known frequency):
%    \begin{align}
%    var(\hat{A}) &\geq CRB(A)\approx \frac{2\sigma^2}{N}\\
%    var(\hat{B}) &\geq CRB(B)\approx \frac{2\sigma^2}{N}\\
%    var(\hat{C}) &\geq CRB(C)\approx \frac{\sigma^2}{N}
%    \end{align}
%  \item The CRLB for the four parameter model (unknown frequency):
%  \begin{align}
%    var(\hat{A})      &\geq CRB(A)     \approx  \frac{2\sigma^{2}}{N}(1+\frac{3B^2}{\alpha^2})\\
%    var(\hat{B})      &\geq CRB(B)     \approx  \frac{2\sigma^{2}}{N}(1+\frac{3A^2}{\alpha^2})\\
%    var(\hat{C})      &\geq CRB(C)     \approx  \frac{\sigma^2}{N}\\
%    var(\hat{\omega}) &\geq CRB(\omega)\approx  \frac{24\sigma^2}{\alpha^2N^3}
%    \end{align}
%\end{itemize}
%\end{frame}

%\subsection{Cramer-Rao Bound on Leakage Parameter}
%\begin{frame}{CRLB on L}
%\begin{itemize}
%  \item The CRLB of L, by using the transformation of parameters,
%    \begin{align}
%    CRB(L)&=\left[\frac{\partial L}{\partial \mathbf{\theta}}\right] CRB(\mathbf{\theta})\left[\frac{\partial L}{\partial \mathbf{\theta}}\right]^{T}
%    \end{align}where
%    \begin{align}
%    \left[\frac{\partial L}{\partial \mathbf{\theta}}\right]& =\left[-\frac{4AC^2}{\alpha^{4}}~~-\frac{4BC^2}{\alpha^{4}} ~~\frac{4C}{\alpha^{2}} ~~0\right]
%    \end{align}
%    Therefore \begin{align}
%    var(L)& = CRB(L) = {2\sigma^2}\left(\frac{16C^4}{\alpha^{6}N}+\frac{8C^2}{\alpha^{4}N}\right)
%    \end{align}
%  \item [NOTE:] CRB(L) obtained by using the three parameter model and by using the four parameter model are same because L is independent of frequency.
%  %\item The $L=\frac{2C^2}{\alpha^{2}}$ is non-linear, therefore the estimator $\hat{L}$ is asymptotically efficient as $N\rightarrow\infty$.
%\end{itemize}
%\end{frame}

%\subsection{Simulation Results}
%\begin{frame}{Monte Carlo simulations}
%\begin{itemize}
%  \item The performance of parameter estimations are evaluated by the Monte-Carlo simulations using the empirical mean-squared error criteria,
%  \begin{align}
%  EMSE &=\frac{1}{M}\sum_{m=0}^{M-1} ( \hat{L}_{m}-\mathrm{E} ( \hat{L}_{m}) )
%    \end{align}where M is number of independent realization.
%  \item Experimental conditions:
%        \begin{align}
%        M&=2000\\
%        L&\ll(A^2+B^2)\\
%        SNR&\gg 20~dB
%         \end{align}
%\end{itemize}
%\end{frame}
%
%\subsection{Simulation Results}
%\begin{frame}{Simulation Results}
%\begin{figure}[!t]
%    %\centering
%    \raggedright
%    \psfrag{b}[c][c][0.5][0]{Logarithmic Scale (in dB)}
%    \psfrag{a}[c][c][0.5][0]{SNR (in dB)}
%    \subfloat[Signal-to-noise ratio]{\label{fig:gull}{\psfig{figure=CRLB_&_EMSE_vs_SNR_for_different_N,width=4cm, height=4cm}}}
%    \psfrag{b}[c][c][0.5][0]{Logarithmic Scale (in dB)}
%    \psfrag{a}[c][c][0.5][0]{N}
%    \subfloat[Number of  samples]{\label{fig:tiger}{\psfig{figure=CRLB_&_EMSE_vs_N.eps,width=4cm, height=4cm}}}
%    \psfrag{b}[c][c][0.5][0]{Logarithmic Scale (in dB)}
%    \psfrag{a}[c][c][0.5][0]{L (in dB)}
%    \subfloat[Leakage parameter]{\label{fig:mouse}{\psfig{figure=CRLB_&_EMSE_vs_L_for_different_N,width=4cm, height=4cm}}}
%	%\vspace{-0.5cm}
%    \caption{Simulation Results}
%\end{figure}
%\end{frame}
%
%\subsection{Simulation Results}

%\subsection{Leakage parameter}
%\begin{frame}{Leakage parameter}
%\begin{figure}[!t]
%    \psfrag{b}[c][c][0.7][0]{Logarithmic Scale (in dB)}
%    \psfrag{a}[c][c][0.7][0]{L (in dB)}
%    \vspace{-0.6cm}
%    \centerline{\psfig{figure=CRLB_&_EMSE_vs_L_for_different_N,width=6cm, height=6cm}}
%	\vspace{-0.5cm}
%    \caption{The CRLB and the EMSE as the function of L.}
%	\label{fig:L_plot}
%\end{figure}
%\end{frame}
%
%\subsection{Signal-to-noise ratio}
%\begin{frame}{Signal-to-noise ratio}
%\begin{figure}[!t]
%    \psfrag{b}[c][c][0.7][0]{Logarithmic Scale (in dB)}
%    \psfrag{a}[c][c][0.7][0]{SNR (in dB)}
%    \vspace{-0.6cm}
%    \centerline{\psfig{figure=CRLB_&_EMSE_vs_SNR_for_different_N,width=6cm, height=6cm}}
%	\vspace{-0.5cm}
%    \caption{The CRLB and the EMSE as the function of SNR.}
%	\label{fig:SNR_plot}
%\end{figure}
%\end{frame}
%
%\subsection{Number of  samples}
%\begin{frame}{Number of samples}
%\begin{figure}[!t]
%    \psfrag{b}[c][c][0.7][0]{Logarithmic Scale (in dB)}
%    \psfrag{a}[c][c][0.7][0]{N}
%    \vspace{-0.4cm}
%    \centerline{\psfig{figure=CRLB_&_EMSE_vs_N.eps,width=6cm, height=6cm}}
%	\vspace{-0.5cm}
%    \caption{The CRLB and the EMSE as the function of N.}
%	\vspace{-0.3cm}
%\label{fig:N_plot}
%\end{figure}
%\end{frame}
%\section{}
%\begin{frame}{References}
%\begin{itemize}
%  \item[1] \emph{IEEE Standard for digitizing waveform recorders}, IEEE Standard 1057, 1994.\vspace{3mm}
%  \item[2] S. M. Kay, \emph{Fundamentals of Statistical Signal Processing: Estimation Theory}, Prentice Hall, Upper Saddle River, New Jersey 07458, 1993.\vspace{3mm}
%  \item[3] A. Nehorai and B. Porat, ``Adaptive comb filtering for harmonic signal enhancement", \emph{Trans. on Acoustics, Speech and Signal Processing}, 34(5) 1124�1138, Oct 1986.
%\end{itemize}
%\end{frame}


