%----------------------------------------------------------------------------------------
% Curriculum vitae of Pravin Kumar Rana
%
% possible options include font size ('10pt', '11pt' and '12pt')
% paper size ('a4paper', 'letterpaper', 'a5paper', 'legalpaper', 'executivepaper' and 'landscape')
% font family ('sans' and 'roman')
%----------------------------------------------------------------------------------------
\documentclass[10pt, a4paper, sans]{moderncv}

%----------------------------------------------------------------------------------------
% CV THEME - options include: 'casual' (default), 'classic', 'oldstyle' 
% and 'banking'
%----------------------------------------------------------------------------------------
\moderncvstyle{classic} 

%----------------------------------------------------------------------------------------
% CV COLOR - options include: 'blue' (default), 'orange', 'green', 'red', 
% 'purple', 'grey' and 'black'
%----------------------------------------------------------------------------------------
\moderncvcolor{blue} 

%----------------------------------------------------------------------------------------
% CV GEOMETRY 
%----------------------------------------------------------------------------------------
\usepackage[scale=.8, top=2cm,  bottom=2cm]{geometry} % Reduce document margins
\setlength{\hintscolumnwidth}{2cm} % Uncomment to change the width of the dates column
%\setlength{\textheight}{1800pt}
\setlength{\footskip}{1in}
%\usepackage{parskip}\setlength{\parskip}{1cm}
\usepackage{etoolbox}
%\patchcmd{\subsection}{\vspace*{200cm}}{\vspace{10ex}}{}{}
\usepackage[ngerman,english]{babel}
\usepackage{lipsum}

%----------------------------------------------------------------------------------------
% CV SYMBOLS and ENCODING
%----------------------------------------------------------------------------------------
\usepackage{graphicx, wasysym, fontawesome}% for symbols
\usepackage[utf8]{inputenc}% character encoding

%----------------------------------------------------------------------------------------
% BIBLIOGRAPHY - MULTIPLE ENTRIES   
%----------------------------------------------------------------------------------------
\usepackage[resetlabels]{multibib}
\newcites{journal,conferences,patentsissued,technical,abstracts,theses,patentspublished,invitedtalk,technicaldemo,conferencetalk,preprints}{{Refereed journal articles},{Refereed conference articles},{Issued},{Technical reports},{Conference abstracts},{Dissertations},{Published},{Invited},{Technical Demo},{Conference Special Session},{Preprints}}

\makeatletter
\renewcommand*{\bibliographyitemlabel}{\@biblabel{\arabic{enumiv}}}
\renewcommand*\httplink[2][]{{\urlstyle{sf}\expandafter\href#2}}
\makeatother

%----------------------------------------------------------------------------------------
% FIRST AND LAST NAME 
%----------------------------------------------------------------------------------------
\firstname{\huge \color{color1}{Pravin Kumar Rana}
{\color{color1}\rule{\textwidth}{4pt}}} \familyname{}

%----------------------------------------------------------------------------------------
% START OF CONTENT
%----------------------------------------------------------------------------------------
\begin{document}
\maketitle \vspace{-18mm}

%----------------------------------------------------------------------------------------
% CONTACT INFORMATION _{OK}
%----------------------------------------------------------------------------------------
\section{Contact Information}
\cvcomputer{\faGlobe}{\href{https://www.algopra.com}{www.algopra.com}}{\faAt}{pravinkumarrana@gmail.com}
\cvcomputer{\faHome}{\"{O}stersk\"{a}rsv\"{a}gen 16A, 18453 Sweden}{\faPhone}{+46~(0)~738~790~621}

%----------------------------------------------------------------------------------------
%	TITLE
%----------------------------------------------------------------------------------------
% Experienced computer vision engineer specialized in agile methodology.
% Agile Computer Vision and Image Processing Expert

%----------------------------------------------------------------------------------------
%	OBJECTIVE
%----------------------------------------------------------------------------------------
%\section{Objective}
%\cvitem{}{To conduct theoretically sound and application motivated research in an industrial environment}%academic}

%----------------------------------------------------------------------------------------
% SUMMARY
%----------------------------------------------------------------------------------------
%
\section{Summary}
\cvlistitem{10+ years of experience in embedded technology encompassing product research and development, agile project and team management, and algorithm development.}
%\cvlistitem{10+ years of experience in embedded product development, agile project and team management, innovation, and research in the fields of computer vision, image and video processing, and machine learning}
\cvlistitem{20+ granted and pending US patents, along with 10+ publications in referred journals and conferences in areas including gaze and eye tracking systems, stereoscopic and multi-view video, and image data processing.}%\cvlistitem{20+ granted and pending US patents, along with 10+ publications in referred journals and conferences, demonstrating innovative problem-solving within areas such as gaze and eye tracking systems, stereoscopic and multi-view video, and image data processing.}
%
%----------------------------------------------------------------------------------------
% EXPERIENCE SECTION
%----------------------------------------------------------------------------------------
% accomplished [X] as measured by [Y], by doing [Z].?
%
\section{Experiences}
%
\subsection{Tobii AB, Danderyd, Sweden}
%
\cventry{2020 -- \textcolor{white}{2020}}{Product Owner}{XR Segment} {}{}{} 
\cvlistitem{Leading a multidisciplinary team of engineers using Agile methodology to the end-to-end development of Tobii's next-generation off-axis eye tracking platform for XR, ensuring delivery of a high-quality platform that meets specified business and customer values within time and cost constraints.} 
\cvlistitem{Delivered exceptional value to tier-1 tech companies by skillfully tailoring Tobii XR eye tracking solutions to precisely match customer needs, timeline, and provide meaningful customer-centric results}
%
\cventry{2016 -- 2020}{Senior Algorithm Developer}{XR segment} {}{}{} % 2014/03/31
\cvlistitem{Led a team of six or more computer vision algorithm developers as a technical lead and scrum master that successfully developed 
%Tobii EyeCore\textsuperscript{\tiny TM} 
algorithms for Tobii VR4 eye tracking platform and performed custom adaptations %of the algorithms 
to ensure a seamless integration of  the platform into virtual reality headsets from various vendors, including \textbf{Tobii HTC Vive Development Kit, HTC Vive Pro Eye, StarVR One, Pico Neo 2 Eye, HP Reverb G2 Omnicept Edition, and Qualcomm Snapdragon Platforms}. Increased tier-one technology customer traction and sales were key indicators of the success of this project.}
\cvlistitem{Led end-to-end development and deployment of Tobii's VR4 eye-tracking platform's optical system calibration software for a tier-1 tech customer.}%\cvlistitem{Oversaw the end-to-end development of Tobii's VR4 eye-tracking platform's optical system calibration software, collaborating with computer vision developers, verifying adherence to specifications, conducting rigorous in-house testing, and supervising the successful deployment at a tier-1 tech customer production facility.}
\cvlistitem{Enhanced customer value by leading VR headset eye tracking integration projects, optimizing timelines, resources, and deliverables. Provided technical insights to engineering teams and collaborated with stakeholders to prioritize product focus.}%\cvlistitem{Additionally, optimized customer value through critical strategic decision-making for eye tracking integration projects in virtual reality headsets, managing timelines, resources, requirements, and deliverables from vendors, while providing technical feedback to engineering teams and coordinating with product owners and managers to prioritize product stories and future focus areas}
%
\cventry{2014 -- 2016}{Senior Algorithm Developer}{PC Segment} {}{}{} 
\cvlistitem{Developed, enhanced, and customised to seamlessly integrated eye tracking algorithms for Tobii IS4 and IS5 platforms across a diverse range of devices, including \textbf{Tobii EyeX, SteelSeries Sentry, PCEye Mini, Eye Tracker 4C, Acer Predator 21X, Acer Aspire Nitro V 17, and Dell Alienware 17}, resulting in optimised product performance and improved user experience.}
%\cvlistitem{Developed, customised, and improved  Tobii EyeChip\textsuperscript{\tiny{TM}} algorithms for Tobii IS4 and IS5 eye tracking platforms that integrated into various devices such as \textbf{Tobii EyeX, SteelSeries Sentry, PCEye Mini, Eye Tracker 4C, Acer Predator Z301CT,  Acer Predator 21X, Acer Nitro V 17,  and Dell Alienware 17}. Contributed to innovative solutions that optimised product performance and significantly improved user experience, playing a crucial role in the platform's growth and hence company success.} 
%Tobii IS4 eye tracking platform  - Tobii's 6th generation eye tracking platform that developed to meet integration requirements for consumer devices
%Tobii EyeChip - The first ever SoC ASIC specifically engineered for eye tracking.
%
\subsection{KTH Royal Institute of Technology, Stockholm, Sweden}% 01-09-2008 -- 30-03-2014
\cventry{2008 -- 2014}{Graduate Researcher}{ACCESS Linnaeus Center, School of Electrical Engineering} {}{}{}
\cvlistitem{Proposed innovative techniques for enhancing the 3D free-viewpoint TV experience as part of a joint KTH-Ericsson project. Introduced a structured depth-image-based rendering method and improved natural scene geometry estimation, resulting in a significant up to 4 dB improvement in rendered view quality. Recognised through ISO/IEC MPEG standardization report, 2 US patents, 2 peer-reviewed journal articles, and 7 peer-reviewed international conference proceeding}
%\cvlistitem{As part of a joint project between KTH and Ericsson, proposed innovative techniques that significantly enhanced the 3D free-viewpoint TV natural scene viewing experience. By introducing a structured depth-image-based rendering method and improving natural scene geometry estimation through inter-view consistency information among multiview video and depth maps, the rendered view quality improved by up to 4 dB. These contributions were recognized through an ISO/IEC MPEG standardization report, 2 US patent grants, 2 peer-reviewed journal articles, and 7 international conference proceedings.}
\cvlistitem{Assisted with teaching and mentoring undergraduate and graduate students in multimedia signal processing, image and video processing, and information theory and source coding courses.}
%
%\vspace{1mm}
\subsection{Government Polytechnic, Koderma, India}
\cventry{2005 -- 2006}{Lecturer}{Department of Physics}{}{}{}%September 15, 2005 -- July 12, 2006
\cvlistitem{Conducted lectures, laboratory sessions, and exams for approximately 50 introductory and upper diploma level students.}
%
%----------------------------------------------------------------------------------------
%	EDUCATION SECTION
%----------------------------------------------------------------------------------------
%
\section{Education}
%
%\subsection{\href{http://www.kth.se/en}{KTH Royal Institute of Technology, Stockholm, Sweden}}
%\cvitem{2008 -- }{\textbf{Doctor of  Philosophy in Telecommunications}}
%\cvitem{Advisor}{\href{http://www.ee.kth.se/~mflierl/}{\textbf{Markus Flierl}}}
%\cvitem{Research area}{Multi-view and 3D Video Processing}
%\cvitem{Expected}{October 2018}
%\cvitem{Dissertation title}{Mutiview depth imagery enhancement techniques for free-viewpoint television (tentative).}
%
\subsection{\href{http://www.iitkgp.ac.in/}{Indian Institute of Technology, Kharagpur, India}}
\cventry{2006 -- 2008}{Master of Technology in Earth System Science and Technology}{}{}{}{}
%\cvitem{Advisor}{\textbf{Mihir Kumar Dash}}
\cvitem{Grade}{9.53/10.00 Cumulative Grade Point Average}
\cvitem{Specialization}{Satellite Oceanography}
\cvitem{Dissertation}{\textit{Prediction of Antarctic sea ice edge using active contour model}.}
\cvlistitem{Employed satellite imagery and gradient vector flow to accurately model daily and monthly variations of the sea ice edge in Antarctica, achieving up to 90\% accuracy.}
%
%\vspace{1mm}
\subsection{\href{http://ranchiuniversity.org.in/}{Ranchi University, Ranchi, India}}
\cventry{2002 -- 2004}{Master of Science in Physics}{}{}{}{}
%\cvitem{Advisor{\textbf{K. K. Dey}}
\cvitem{Grade}{Graduated with first class first (71.81\%)}
\cvitem{Specialization}{Electronics and Communication}
\cvitem{Dissertation}{\textit{Study on general impedance converter and its application in realisation of second order active electronic resistor-capacitor-filters}.}
\cvlistitem{Implemented operational amplifier based general impedance converter circuit and used it to simulate different types of second-order active electronic resistor-capacitor filters}
%
%\vspace{1mm}
\cventry{1998 -- 2002}{Bachelor of Science (Honours) in Physics}{}{}{}{}
\cvitem{Grade}{Graduated with first class (74.75\%)}
%
%----------------------------------------------------------------------------------------
%	CERTIFICATIONS_{OK}
%----------------------------------------------------------------------------------------
%
\section{Certifications}
\subsection{\href{https://www.coursera.org/account/accomplishments/specialization/certificate/P7NSQW4WCV8D}{deeplearning.ai}}
\cventry{2018}{Deep Learning Specialization}{Andrew Ng}{Cousera}{}{}
\subsection{\href{https://www.coursera.org/account/accomplishments/verify/48PZC9AAWZYH}{University of Michigan}}
\cventry{2018}{Programming for Everybody}{Charles Severance}{Cousera}{}{}
\subsection{\href{https://www.scrumalliance.org/community/profile/prana23}{Scrum Alliance}}
\cventry{2017}{Certified Scrum Master}{}{}{}{}%$\#$000686203
\subsection{\href{http://www.iitkgp.ac.in/}{Indian Institute of Technology, Kharagpur, India}}
\cventry{2007}{Computer Network Management}{}{}{}{}
%
%----------------------------------------------------------------------------------------
% COURSES
%----------------------------------------------------------------------------------------
%
%\section{Courses}
%\subsection{KTH Royal Institute of Technology, Stockholm, Sweden}
%\cvitem{}{F2E5316 Information Theory, EN2500 Information Theory and Source Coding, FEL3300 Convex Optimization with Engineering Applications, EQ2800 Optimal Filtering, F2E5320 Matrix Algebra, FDD3315 Multiple View Geometry, FEO3110 Science Communication, FDS3102 Writing Scientific Articles, FEM3210 Estimation Theory, DD3364  Elements of Statistical Learning}
%\subsection{Indian Institute of Technology, Kharagpur, India}
%\cvitem{}{CL60014 Satellite Oceanography (Specialization), CL60001 Dynamics of Fluvial Systems, CL60003 Advance Meteorology, CL60002 Ocean Dynamics, CL60002 Global Tectonics and Climate, CL60006 Ocean Color and Applications, CL60013  Polar Science, CL60005 Planetary and Marine Boundary Layer, CL60015 Modelling of Extreme Events, CL69002 Ocean and Storm Surge Modelling (Laboratory), CL69003 Data Analysis and GIS Applications (Laboratory), CL69001 Atmospheric and Hydrological Modelling (Laboratory)}
%\subsection{Ranchi University, Ranchi, India}
%\cvitem{}{Electronics and Communications (Specialization), Computer Application in Physics, Electronics, Quantum Mechanics, Solid State Physics, Atomic, Molecular and Statistical Physics, Optics Experiments (Laboratory), Electrical and Electronics Experiment (Laboratory), Computer Application in Physics (Laboratory)}

%----------------------------------------------------------------------------------------
%	PROJECTS
%----------------------------------------------------------------------------------------
%\section{Projects}
%\subsection{\href{http://www.tobii.com}{Tobii AB, Danderyd, Sweden}}
%\cvitem{2016 -- present}{\textbf{{VR eye tracking platform and integrations}}}
%\cvitem{}{\textbf{Products}: Tobii HTC Vive Development Kit, HTC Vive Pro Eye, StarVR One, Qualcomm Snapdragon 845 Mobile VR Platform, Pico Neo 2 Eye}
%\cvitem{Responsibilities}{Scrum Master, Technical Lead, and Senior Algorithm Developer}
%
%\cvitem{2015 -- 2016}{\textbf{{IS4 eye tracking platform}}}
%\cvitem{}{Tobii's 6th generation eye tracking platform developed to meet integration requirements for consumer devices with the first ever SoC ASIC specifically engineered for eye tracking - Tobii EyeChip$^{\texttt{\tiny{TM}}}$}
%\cvitem{}{\textbf{Products}: Tobii PCEye Mini, Tobii Eye Tracker 4C, Alienware 17, Acer Predator Z301CT, Acer Predator 21X, Acer Aspire V 17 Nitro Black Edition}
%\cvitem{Responsibilities}{Senior Algorithm Developer}
%%https://www.tobii.com/group/news-media/press-releases/2016/4/tobii-dynavox-launches-next-generation-eye-tracking-devices-based-on-the-tobii-is4-platform/
%% https://www.tobii.com/group/news-media/press-releases/2017/1/beyond-gaming-acer-launches-consumer-notebook-with-tobii-eye-tracking/
%%https://www.tobii.com/group/news-media/press-releases/2016/9/tobii-receives-order-from-alienware-regarding-the-is4-eye-tracking-platform/
%
%\cvitem{2014 -- 2015}{\textbf{{IS3 eye tracking platform}}}
%\cvitem{}{Tobii's 5th generation eye tracking platform developed for integration into consumer devices such as monitors and computer peripherals}
%\cvitem{}{\textbf{Products}: Tobii EyeX, SteelSeries Sentry, MSI GT72S Dominator}
%\cvitem{Responsibilities}{Senior Algorithm Developer}

%\subsection{\href{https://www.kth.se/en}{KTH Royal Institute of Technology - Telefonaktiebolaget L M Ericsson (publ), Stockholm, Sweden}}
%\cvitem{2008 -- 2011}{\textbf{{3D video coding}}}
%\cvitem{}{KTH-Ericsson joint research to develop new techniques for 3D video coding to facilitate free viewpoint viewing that are relevant to future standardization efforts and to broaden the range of expertise in 3D video}
%\cvitem{}{\textbf{Results}: One MPEG 3DTV report, Two US patent grants, Three refereed conference articles}
%\cvitem{Contributions}{
%\begin{itemize}
%\item Designed multiview depth image enhancement algorithm by exploiting the color and depth association through variational Bayesian inference color classifier to improve rendering quality
%\item Generated a noisy volumetric depth confidence of a natural scene by inconsistent depth estimates from multiple viewpoints and adaptively denoised the estimates by 3D wavelet thresholding
%\item Developed multiview depth consistency testing to address inter-view inconsistencies in estimated depth maps at various viewpoints for depth-image based rendering
%\item Devised inter-view consistency adaptive view synthesis to enrich the free-viewpoint experience by improving the visual quality of view synthesis at any arbitrary viewpoint
%\item Addressed the limited free-viewpoint television transmission requirements by proposing structured depth maps, a consistent depth representation technique for dynamic natural scenes
%\end{itemize}}
%\cvitem{Leader}{\textbf{Markus Flierl}}
%\cvitem{Industrial partner}{\href{https://labs.ericsson.com/}{Ericsson AB, Stockholm, Sweden}}

%\cvitem{2016 -- Present}{\textbf{{Tobii eye tracking in virtual reality}}}
%\cvitem{Description}{Implemented novel and unique solutions and performed custom adaptations and optimisation of Tobii's EyeCore\textsuperscript{\tiny TM} eye tracking algorithms for the virtual reality head mounted displays from a number of different vendors and manufactures}
%\cvitem{Products}{Tobii eye tracking in StarVR, Tobii eye tracking VR development kit for the HTC Vive and the Qualcomm Snapdragon 845 mobile VR platform}
%\cvitem{Role and responsibilities}{Scrum Master}
%\cvitem{Contributions}{Personal eye parameters calibration, Image quality, glint detector, }

%\subsection{KTH Royal Institute of Technology, Stockholm, Sweden}
%\cvitem{2008 -- 2014}{Designing and improving depth image based rendering algorithms for high quality free-viewpoint television experience.}
%\cvitem{}{$\bullet$ Designed multiview depth image enhancement algorithm by exploiting the color and depth association through variational Bayesian inference color classifier to improve rendering quality.}
%\cvitem{}{$\bullet$ Generated a noisy volumetric depth confidence of a natural scene by inconsistent depth estimates from multiple viewpoints and adaptively denoise this volume by 3D wavelet thresholding.}
%\cvitem{}{$\bullet$ Developed multiview depth consistency testing to address inter-view inconsistencies in estimated depth maps at various viewpoints for depth-image based rendering.}
%\cvitem{}{$\bullet$ Devised inter-view consistency adaptive view synthesis to enrich the free-viewpoint experience by improving the visual quality of view synthesis at any arbitrary viewpoint.}
%\cvitem{}{$\bullet$ Addressed the free-viewpoint television transmission requirements by proposing structured depth maps, a consistent depth representation technique for dynamic natural scenes.}

%----------------------------------------------------------------------------------------
% TEACHING EXPERIENCE
%----------------------------------------------------------------------------------------
%\section{Teaching Experience}
%\subsection{KTH Royal Institute of Technology, Stockholm, Sweden}
%\cventry{2008 -- 2014}{Teaching assistant}{School of Electrical Engineering} {}{}{}
%\cvitem{}{Image and Video Processing}{}{}{}{}
%\cvitem{}{Project Course on Multimedia Signal Processing}{}{}{}{}
%\cvitem{}{Information Theory and Source Coding}{}{}{}{}
%\cvitem{}{Degree Project in Electrical Engineering}{}{}{}{}
%\subsection{Government Mining Technical Institute, Koderma, India}
%\cventry{2005 -- 2006}{Lecturer}{Department of Physics}{}{}{}%
%\cvitem{}{Physics}{}{}{}{}

%----------------------------------------------------------------------------------------
% PROFESSIONAL ACTIVITIES
%----------------------------------------------------------------------------------------
%\section{Professional Activities}
%\subsection{Affiliations}
%\cvitem{2017 -- present}{{Scrum Alliance}}{}{}{}{}
%\cvitem{2013 -- present}{{IEEE Signal Processing Society (IEEE SPS)}}{}{}{}{}
%\cvitem{2011 -- present}{{European Association for Signal Processing (EURASIP)}}{}{}{}{}
%\cvitem{2010 -- present}{{Institute of Electrical and Electronics Engineers (IEEE)}}{}{}{}{}
%\subsection{Reviewer}
%\cvitem{2016 -- present}{IEEE International Conference on Acoustics, Speech, and Signal Processing (ICASSP)}{}{}{}{}
%\cvitem{2015 -- present}{IEEE Signal Processing Letters (IEEE SPL)}{}{}{}{}
%\cvitem{2015 -- present}{IEEE International Conference on Multimedia and Expo (ICME)}{}{}{}{}
%\cvitem{2013 -- present}{IEEE SPS International Workshop on Multimedia Signal Processing (MMSP)}{}{}{}{}
%
%----------------------------------------------------------------------------------------
% ACHIEVEMENTS
%----------------------------------------------------------------------------------------
%
\section{Achievements}
\cvitem{2021 -- 2023}{Recognised as a top performer at Tobii AB for consistently delivering exceptional results as a team member.}
\cvitem{2016 -- 2023}{Received \textbf{20 US patent grants} as an inventor and co-inventor in the fields of gaze and eye tracking optical systems, stereoscopic and multi-view video, and image data representation and processing.}
%\cvlistitem{Exhibited adeptness in identifying and resolving technical challenges, evidenced by 20+ granted and pending US patents as both an inventor and co-inventor in the fields of gaze and eye tracking optical systems, stereoscopic and multi-view video, and image data representation and processing.}
\cvitem{2006 -- 2008}{Received the prestigious \textbf{Ministry of Human Resource Development Scholarship} from the Government of India to pursue a Master of Technology degree at \textbf{Indian Institute of Technology, Kharagpur}, India.}
\cvitem{2006}{Achieved a top 5\% rank (254 out of 4904) in India's competitive \textbf{Graduate Aptitude Test in Engineering,} an assessment by the Indian Government evaluating comprehensive understanding of engineering and science subjects for Master's Program admissions.}
\cvitem{2004}{Secured the top position in the Master of Science examination at Ranchi University, Ranchi, India.}
\cvitem{2002}{Ranked third in the Bachelor of Science examination from Ranchi University, Ranchi, India.}
%
%----------------------------------------------------------------------------------------
% SKILLS 
%----------------------------------------------------------------------------------------
%
\section{Competencies}
%
\subsection{Management}
\cvcomputer{Methodology}{Agile Management}{Tool}{Jira, Git, Confluence, Microsoft Office}
%
\subsection{Technical}
\cvcomputer{Platform}{Linux, Windows, macOS}{Libraries}{OpenCV, Python Imaging, Eigen}
\cvcomputer{Programming}{C++, Python}{Environments}{Eclipse, Visual Studio, CLion}
%\cvitem{Webmaster}{ACCESS Linnaeus Centre, KTH Royal Institute of Technology, Stockholm, Sweden}

%----------------------------------------------------------------------------------------
% PATENTS AND PUBLICATIONS
%----------------------------------------------------------------------------------------
%\newpage 
\section{Patents}
\nocitepatentsissued{*}
\bibliographystylepatentsissued{ieeetr}
\bibliographypatentsissued{patentsissued}
\vspace{1mm}
\nocitepatentspublished{*}
\bibliographystylepatentspublished{ieeetr}
\bibliographypatentspublished{patentspublished}
\vspace{1mm}
\section{Publications}
\nocitepreprints{*}
\bibliographystylepreprints{IEEEtran}
\bibliographypreprints{preprints}
\vspace{1mm}
\nocitejournal{*}
\bibliographystylejournal{IEEEtran}
\bibliographyjournal{journal}
\vspace{1mm}
\setcounter{enumiv}{0}
\nociteconferences{*}
\bibliographystyleconferences{ieeetr}
\bibliographyconferences{conferences}
\vspace{1mm}
\setcounter{enumiv}{0}
\nocitetechnical{*}
\bibliographystyletechnical{ieeetr}
\bibliographytechnical{technical}
\vspace{1mm}
\setcounter{enumiv}{0}
\nociteabstracts{*}
\bibliographystyleabstracts{ieeetr}
\bibliographyabstracts{abstracts}
\vspace{1mm}
\setcounter{enumiv}{0}
\nocitetheses{*}
\bibliographystyletheses{ieeetr}
\bibliographytheses{theses}
%
%----------------------------------------------------------------------------------------
% CONFERENCE, WORKSHOP, INVITED TALKS 
%----------------------------------------------------------------------------------------
%
\section{Talks} 
\cvitem{2019}{\textbf{\textit{Eye tracking in 5G era}}, Tobii Develop Beyond, Tobii AB, Stockholm, Sweden}
\cvitem{2017}{\textbf{\textit{Tobii eye tracking VR development kit}}, ACCESS Data Analytics Workshop, KTH Royal Institute of Technology, Stockholm, Sweden}
\cvitem{2015}{\textbf{\textit{Stereo vision based distance estimation in Tobii IS4 eye tracking platform}}, Tobii Develop Beyond, Tobii AB, Stockholm, Sweden}
\cvitem{2014}{\textbf{\textit{Statistical methods for inter-view depth enhancement}}, 3DTV Conference, Special Session on Free-Viewpoint TV and Related Technologies, Budapest, Hungary}
\cvitem{2013}{\textbf{\textit{Multiview depth map enhancement by variational Bayes inference estimation of Dirichlet mixture models}}, IEEE International Conference on Acoustics, Speech, and Signal Processing, Vancouver, Canada}
%\cvitem{2013}{ACCESS Industrial Workshop, KTH Royal Institute of Technology, Stockholm, Sweden}
\cvitem{2012}{\textbf{\textit{A variational Bayesian inference framework for multiview depth Image enhancement}}, IEEE International Symposium on Multimedia, Irvine, California, USA}
\cvitem{2012}{\textbf{\textit{Denoising of volumetric depth confidence for view rendering}}, 3DTV-Conference, Zurich, Switzerland}
\cvitem{2012}{\textbf{\textit{Depth pixel clustering for consistency testing of multiview depth}}, European Signal Processing Conference, Bucharest, Romania}
\cvitem{2012}{\textbf{\textit{Depth pixel clustering for consistency testing of multiview depth}}, ACCESS PhD and Post-doc Workshop, KTH Royal Institute of Technology, Stockholm, Sweden}
%\cvitem{2011}{ACCESS Industrial Workshop, KTH Royal Institute of Technology, Stockholm, Sweden}
\cvitem{2011}{\textbf{\textit{View interpolation with structured depth from multiview video}}, European Signal Processing Conference, Barcelona, Spain}
\cvitem{2010}{\textbf{\textit{Depth consistency testing for improved view interpolation}}, IEEE International Workshop on Multimedia Signal Processing, Saint Malo, France}
\cvitem{2008}{\textbf{\textit{Prediction of Antarctic sea ice edge using artificial intelligence}}, SCAR/IASC IPY Open Science Conference, St. Petersburg, Russia}
\cvitem{2008}{\textbf{\textit{Prediction of sea ice edge by using image processing techniques}}, Department of Wind Energy, Technical University of Denmark (DTU),  Risø, Denmark}
%
%----------------------------------------------------------------------------------------
% MEDIA 
%----------------------------------------------------------------------------------------
%
\section{Media Coverage}
\cvitem{[1]}{A. Wahlberg and J. Koch, ``\textbf{\textit{Free Viewpoint Television: Det nya TV-tittandet}}'', {Osqledaren}, (K\r{a}rhuset Nymble, Drottning Kristinas v\"{a}g 19, THS, Stockholm), no. 3, pp. 10 -- 11, 2009/2010. (Swedish)}
%
%----------------------------------------------------------------------------------------
% LANGUAGES
%----------------------------------------------------------------------------------------
%
%\section{Languages}
%\cvitem{Native}{Hindi}
%\cvitem{Fluent}{English}
%
%----------------------------------------------------------------------------------------
% INTERESTS
%----------------------------------------------------------------------------------------
%\section{Interests}
%\cvitem{}{Photography}
%\cvitem{Fluent}{English}

%----------------------------------------------------------------------------------------
% REFERENCES 
%----------------------------------------------------------------------------------------
%\section{References}
%\cvitem{}{Available upon request.}

%\subsection{Mark  Ryan}
%\cvitem{}{Line Manager}
%\cvitem{\faHome}{Division of Information Science and Engineering}
%\cvitem{}{Tobii AB (publ.), SE-10044, Stockholm, Sweden}
%\cvitem{\faPhone}{+46 8 790 7425}
%\cvitem{\faAt}{mark.ryan@tobii.com}
%\cvitem{\faGlobe}{\href{http://people.kth.se/~mflierl/}{www.people.kth.se/$\sim$mflierl}}
%
%\subsection{Markus Flierl}
%\cvitem{}{Associate Professor}
%\cvitem{\faHome}{Division of Information Science and Engineering}
%\cvitem{}{KTH Royal Institute of Technology, SE-10044, Stockholm, Sweden}
%\cvitem{\faPhone}{+46 8 790 7425}
%\cvitem{\faAt}{markus.flierl@ee.kth.se}
%\cvitem{\faGlobe}{\href{http://people.kth.se/~mflierl/}{www.people.kth.se/$\sim$mflierl}}
%
%\subsection{Ivana Girdzijauskas}
%\cvitem{}{European Patent Attorney}
%\cvitem{\faHome}{Ericsson AB, Kista, SE-16483, Stockholm, Sweden}
%\cvitem{\faPhone}{+46 761 441 403}
%\cvitem{\faAt}{ivana.girdzijauskas@ericsson.com}
%\cvitem{\faLinkedin}{\href{https://www.linkedin.com/in/ivana-girdzijauskas-8479b53/}{Ivana Girdzijauskas}}
%
%\subsection{Zhanyu Ma}
%\cvitem{}{Assistant Professor}
%\cvitem{\faHome}{Pattern Recognition and Intelligent System Laboratory}
%\cvitem{}{Beijing University of Posts and Telecommunications, 100876 - Beijing, China}
%\cvitem{\faPhone}{+86 1346 632 3341}
%\cvitem{\faAt}{mazhanyu@bupt.edu.cn}
%\cvitem{\faGlobe}{\href{http://www.pris.net.cn/en/introduction-en/teacher-en/zhanyu}{Web: www.iitbbs.ac.in/pcpandey/}}
%
%\subsection{Mihir Kumar Dash}
%\cvitem{}{Assistant Professor}
%\cvitem{\faHome}{Center for Oceans, Rivers, Atmosphere and Land Sciences}
%\cvitem{}{Indian Institute  of  Technology, 721302 - Kharagpur, India}
%\cvitem{\faPhone}{+91 3222 281 824}
%\cvitem{\faMobile}{+91 99 33 078541}
%\cvitem{\faAt}{mihir@coral.iitkgp.ernet.in}
%\cvitem{\faGlobe}{\href{http://www.iitkgp.ac.in/fac-profiles/showprofile.php?empcode=bXmYW}{Web: www.iitkgp.ac.in/mkdash/}}
%
%\vfill
%\center \copyright ~Pravin Kumar Rana
%\center ~Pravin Kumar Rana
%\center{\textbullet~~\faPhone~+46~(0)~738~790~621~~\textbullet~~\faGlobe~{\href{http://www.algopra.com}{www.algopra.com}~~\textbullet~~\faAt~pravinkumarrana@gmail.com~~\textbullet}
%\center Stockholm, \today{}
\end{document}
%----------------------------------------------------------------------------------------
% THE END
%----------------------------------------------------------------------------------------